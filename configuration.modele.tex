% Documentation des macro à définir pour modifier le rapport.
% Les fichiers devant être créés lorsque certaines options sont activées sont situés dans le répertoire `contenu`
% (celui dans lequel se trouve le présent fichier).s
% Les macros marquées comme facultatives ne doivent pas (MUST NOT au sens de la RFC 2119) être définies si elles
% ne sont pas utilisées.

% Décommenter la ligne suivante pour afficher la partie Annexes
% Nécessite la création d'un fichier `annexes.tex`
% \def\rpAnnexes{}

% Si cette macro est définie, le lexique/glossaire est affiché
% Nécessite la création d'un fichier `lexique.tex` constitué d'une succession de `\item[terme]{définition}`
% \def\rpLexique{}

% Si "Glossaire" ne convient pas :
% \def\rpTitreLexique{Lexique}

% Décommenter la partie suivante pour afficher la webographie
% \def\rpWebographie{}

% Filière du (des) rédacteurs
% Prédéfinies : \rpFun, \rpFdeux, \rpFtrois, \rpFquatre et \rpFcinq
\def\rpFiliere{ -- obligatoire -- }

% Année du rédacteur : 1, 2, ou 3
\def\rpAnnee{ -- obligatoire -- }

% Type de rapport (Stage, Projet)
\def\rpType{ -- obligatoire -- }

% Titre du rapport
\def\rpTitre{ -- obligatoire -- }

% Date de la soutenance
\def\rpDateSoutenance{ -- obligatoire -- }

% Durée du projet ou du stage (exemple : 5 mois)
\def\rpDuree{ -- obligatoire -- }

% Nom du (premier) rédacteur
\def\rpNom{ -- obligatoire -- }

% Nom du second rédacteur
\def\rpSecondNom{ -- facultatif -- }

% Pour changer la taille des interlignes
% \def\rpInterligne{1.0}

% Nom du tuteur Isima
\def\rpTuteurIsima{ -- obligatoire -- }

% Nom du tuteur entreprise
\def\rpTuteurEntreprise{ -- obligatoire -- }
