% Partie 1.1 - l'IoT %

\subsection{Les PAN}

Un réseau local (\textbf{LAN} pour \textit{Local Area Network}) est un réseau où les terminaux peuvent communiquer sans avoir besoin d'un accès Internet. Mais il existe un autre type de réseau plus proche de nos contraintes, le réseau personnel (\textbf{PAN} pour \textit{Personal Area Network)}. Dans ce type de réseau le but est de faire aboutir les échanges entre les divers éléments du réseau avec des puissances assez faible pour garantir une autonomie correcte. Ce qui correspond bien à nos besoins avec les capteurs. Ils ont besoins d'être autonome le plus longtemps possible tout en évitant une interaction humaine physique direct. 

Dans le monde des réseau, nous utilisons le terme topologie pour définir l'arrangement des différents nœuds dans un réseau. Avec les réseaux PAN, on se limite assez souvent à trois topologies: maillage complet (\textbf{P2P}), maillage partielle (\textbf{mesh}) et en étoile (\textbf{star}).



Le fait est que comme cette technologie est explorée par plusieurs constructeur et organisme, plusieurs protocoles ont vu le jour, mais sans réel standard, chacun utilise celui qu'il veut. Cela pose pas mal de problème avec l'interopérabilité des éléments dans le réseau qui est pourtant une des caractéristiques phare de l'IoT. Voici une liste succincte de ces divers protocoles.
