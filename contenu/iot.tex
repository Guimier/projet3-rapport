% Partie 1.1 - l'IoT %

\subsection{L'IoT}

\textit{L'Internet of Things}, que nous appellerons maintenant IoT pour le reste de ce rapport, se traduit littéralement comme l'Internet des Objets, mais qu'est-ce-qu'un objet? Dans le monde de l'IoT les objets peuvent se référer à des biens (comme des meubles ou de l'électroménager), des machines, des véhicules, des immeubles ou bien même à quelque chose d'organique comme un être vivant (Homme ou animal), une plante, des sols (pour les cultures).

Alors la question est : Comment pouvons nous tous connecter ? En effet, comment faire en sorte qu'une plante possède un accès réseau. C'est cela que l'IoT veut définir et représente, une connectivité pour tout.

Mais d'abord que veut dire le mot connecter ? Prenons l'exemple d'une chaise, le fait qu'elle soit connectée veut dire que je peux avoir accès à de l'information la concernant, depuis n'importe où, grâce à un accès à Internet, par exemple, est-elle occupée? Si oui, qui est assis dessus? Pour cela nous avons besoin de donner certains attributs de cette chaise, comme un numéro d'identification unique, une manière de la distinguée d'un autre objet. 

Grâce à \textbf{IPv6}, nous pouvons maintenant affecter une adresse unique à tout sans limite réel car l'espace adressable est sans limite pratique (mais il y a bien sur une limite physique). IPv4 est déjà dépassée en terme de capacité d'adressage depuis longtemps mais grâce à des mécanismes comme le NAT/PAT, IPv4 est encore utilisé. Nous reviendrons sur IPv6 un peu plus tard dans ce rapport lors de notre présentation de 6LoWPAN.

Ensuite, nous avons besoins de donner à la chaise un moyen de communiquer avec le monde, soit de manière filaire ou sans fil grâce à des antennes. D'où la nécessiter des protocoles réseaux qui vont devoir transporter les informations, malheureusement ceux que nous utilisons dans la vie tous les jours ne sont pas vraiment adaptées (IPv4, IPv6, Wi-Fi, ...) à l'IoT, de part leur consommation électrique ou de bande-passante. C'est pourquoi de nouveaux protocoles ont vu le jour, spécialement adaptés à ces besoins. Certain se concentre sur la puissance d'émission et d'autre sur la fiabilité dans les milieux bruités, mais tous prennent en compte certaines contraintes de l'IoT.

Aussi, nous parlons d'information et de données mais il faut bien les générer, pour cela nous utilisons différents types de capteurs, comme par example de pression pour savoir si notre chaise est occupé. Un autre type de capteur pourrait être une puce de localisation, ou bien même un capteur d'identification qui pourra nous dire qui est assis sur notre chaise et qui l'a été. De nos jours les capteurs sont extrêmement petits mais ont quand même certaines capacités comme de la mémoire, ce qui est très pratique en cas de coupure temporaire du réseaux, en effet, les données ne sont pas forcément perdue.

\begin{figure}[H]
\begin{center}
\includegraphics[width=15cm]{\rpDossier/images/capteurs.png}
\end{center}
\caption{Exemple de capteurs}
\label{sensors}
\end{figure}

Quelles seront les impacts et les possibilités de l'IoT ? Nous ne sommes limités seulement par notre imagination car nous changeons l'approche de voir et de connecter les objets.

% Monitor %

Prenons quelques exemples pour montrer l'intérêt de l'IoT. Le \textit{monitoring} ne se résume pas aux réseaux et aux machines, nous pouvons aussi l'appliquer pour surveiller l'état d'un patient en médecine. Imaginons quelqu'un avec un problème cardiaque, il possède un pacemaker qui est "connecté", cela veut dire qu'une application sur son téléphone peut dire à cette personne l'état de son cœur, mais aussi à son hôpital. Dans le cas d'une défaillance, une alerte est lancée à l'hôpital qui peut envoyer immédiatement une ambulance, quant à la localisation ils peuvent utiliser un tracker GPS intégré au tout.

Grâce à des algorithmes puissants, nous pourrons même prédire un potentiel problème sans que le patient vienne faire des visites de contrôles régulières, son pacemaker enverra les données pour lui. Avec le nombre de personnes de plus de 65 qui va doubler dans peu de temps, la e-santé et la télé-médecine vont devenir un des plus gros secteurs de l'IoT. 

% Search  %

Les tracas de quotidiens comme la perte de ses clefs ne seront plus un problème, en effet, dans le monde de l'IoT vos clés sont géo-localisables. Cela peut s'appliquer à plein de choses, si ce n'est à toutes, vous ne perdrez plus jamais rien.

% Manage %

Si nous savons où les choses sont et dans quels états elles sont, nous pouvons mieux les manager. Prenons le cas du trafic en ville, si nous savons où les voitures se situent et où elles vont, nous pouvons potentiellement éliminer les bouchons, optimiser les trajets et rediriger les flux plus facilement. Il en va de même pour l'énergie, si nous savons où l'énergie est requise, nous pouvons adapter la production et optimiser les coûts.

% Control %

L'IoT permet aussi de déléguer du contrôle, toujours dans une optique d'optimisation énergétique, nous pouvons déléguer l'heure de départ d'une machine à laver, d'un lave-vaisselle à un contrôleur distant qui lancera le programme au moment où l'énergie sera la moins chère.

% RV + IoT %

Un autre marché que l'IoT va investir est la réalité augmentée (en particulier les jeux vidéo). En effet, en utilisant la réalité augmentée, nous pouvons superposer la réalité (filmée par une caméra) et le monde virtuel grâce à un traitement logiciel. L'IoT va simplement étendre le panel de fonctionnalité de manière infinie en permettant d'interagir avec l'environnement. Par example, il suffirait de lancer son application de réalité augmentée et de toucher l'objet avec lequel nous voulons interagir pour se voir proposer des actions. Le panel d'action se limite seulement à ce que le développeur veut laisser les gens faire.

\begin{figure}[H]
\begin{center}
\includegraphics[width=10cm]{\rpDossier/images/ariot.jpg}
\end{center}
\caption{Réalité augmentée + IoT}
\label{ariot}
\end{figure}

% Privacy %

Le fait est qu'avec l'IoT, la notion de "privé" devient très floue. Nous tenons à évoquer ce point car l'IoT va quelque peut bousculer certaines habitudes. Reprenons notre chaise, comme dit précédemment, nous pouvons facilement savoir qui était assis dessus grâce à notre identifiant unique. Les risques liés à la protection de notre vie privée augmentent avec l'IoT qui collecte agrège des bouts de données liées aux services que l'appareil propose. Le recoupement d'informations peut assez vite convertir des données banales en données personnelles car les événements (action de l'utilisateur) possèdent un lieu, une heure, une récurrence, etc. L'achat régulier de différents types de nourriture peut révéler la religion ou des problèmes de santé. C'est un véritable enjeu lié au Big Data que l'exploitation des données générées par l'IoT. Le volume de données produit va pouvoir permettre aux exploitants de convertir ces données "public" en "privé".

% Security %

Maintenant, il faut aussi se placer du côté de l'exploitant qui va vouloir protéger ses informations, c'est pour cela que la sécurisation des transmissions est l'un des enjeux essentiels de l'IoT. L'inconvénient est que la plus part du temps, les capteurs n'ont aucune puissance de calcul pour crypter l'information, actuellement les capteurs se concentre plus sur l'intégrité du message et à établir une connexion sécurisée. Plus la technologie évoluera, plus la sécurité de l'information se rapprochera de l'appareil, pour ultimement devenir embarquée. 

Il est vrai qu'un simple sniffer réglée sur la bonne fréquence permettra de pouvoir intercepter les informations si celles-ci sont transmisses sans-fil. En effet sans cryptages elles sont envoyées en clair. Cela ne pose pas trop de problème pour des données non-confidentielles mais peut vite devenir problématique  si les informations recueillies sont de nature plus sensibles.

% Data Processing % 

Nous avons parlé précédemment de traitement de l'information, mais aux vues des volumes de données générées ce n'est pas sur un simple pc de bureau que nous pourrons analyser les données. Avec 20 Milliards d'objets connectés en 2020, les datacenters feront face à des charges de travails inédites. Le volume de données traités n'aura jamais était aussi important et ne fera que grossir aux fils des années. Dès aujourd'hui les fournisseurs commencent à désigner leur datacenters en prévision de l'IoT qui arrive à grand pas.

Mais revenons à un niveau plus bas, celui de la transmission entre les capteurs. Nous ne sommes pas encore sur les infrastructures des FAI mais dans notre réseau local. Ce même réseau est celui qui est le plus proche de l'utilisateur. 
