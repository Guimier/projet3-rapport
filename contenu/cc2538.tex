\subsection{Cartes CC 2538}

La première carte que nous avons utilisé est la CC 2538 de \emph{Texas Instruments} (TI), branchée sur le kit de développement \emph{SmartRF 06}.

\subsubsection{Présentation}

\todo[Images et descriptions]

\subsubsection{Test des cartes}

\todo[Copies d’écran RF studio \& description]

\subsubsection{Programmes exemple}

\paragraph{Installation sur les cartes}

\emph{Contiki} fournit un script \emph{Python} pour installer un programme sur une carte CC 2538, appelé \emph{cc2538-bsl} \todo[référence documentation], qui est intégré dans le \texttt{Makefile}.
Nous avons initialement suivi les étapes décrites dans sa documentation pour configurer la communication avec les cartes CC 25358, mais le résultat a invariablement été un message d’erreur indiquant que la carte ne répondait pas aux solicitations du script.

\todo[Copie de l’erreur]

Nous nous sommes alors tournés vers le logiciel développé par TI \todo[nom].
Celui-ci ne fonctionnant que sur un système Windows \todo[Mac aussi ?], nous l’avons installé sur la machine hôte de la machine virtuelle.
Afin de transmettre facilement les fichiers compilés par \emph{Contiki}, nous avons utilisé les outils de partage de dossier proposés par \emph{VirtualBox}.

\todo[Copies d’écran du logiciel]

\paragraph{\texttt{simple-udp-rpl}}

\todo[Description programmes]

\paragraph{Manipulations des LEDs et boutons}

\todo[Description programme]

\paragraph{Développement d’un programme d’action à distance}

\todo[Fonctionnalités, architecture]
