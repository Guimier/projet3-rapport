% Partie 3.3 - Perspectives %

\subsection{Perspectives}

Pour ce projet les perspectives sont multiples, en voici quelques unes :

Dans un premier temps, il serait bien d'arriver à faire fonctionner les Arduino, ne serait ce qu'arriver à transmettre des données entre deux Arduino grâce au programme test fourni avec la pile Arduino 6LoWPAN.

Ensuite, il faudrait modifier ou refaire ce programme pour pouvoir interagir avec les CC 2538, ce qui permettrait de tester l'interopérabilité. Nous pourrions alors voir ma distance couverte par les cartes ce qui permettra de récolter de précieuses informations.

Aussi, pour l'instant les transmissions sont en claire, il serait donc bon de pouvoir ajouter un certain niveau de sécurité via le cryptage des données car on sait que les CC 2538 peuvent supporter l'AES-128.

Ultimement, il serait intéressant de tester l'interconnexion entre deux réseaux 6LoWPAN. Le scénario serait le suivant : une trame part d'un premier réseau 6LoWPAN, sort par le border router, va sur "Internet" (ou un réseau de notre création), puis arrive sur un autre réseau 6LoWPAN.
