% Introduction du rapport

L'Internet des Objets (\textbf{IdO} ou \textbf{IoT} pour \textit{Internet of Things} en anglais) représente la jonction entre Internet et le monde des capteurs qui ne sont généralement pas directement reliés à des équipements actifs des réseaux, comme un routeur par exemple, mais passés plutôt par une phase traitement sur un ordinateur avant d'être potentiellement envoyé sur le réseau.

Tous les concepts derrière l'IoT soulèvent plusieurs problématiques telle que l'accroissement exponentiel du volume de données sur le réseau, dût à l'explosion du nombre d'objet connecté, mais aussi des problèmes de natures énergétiques car les capteurs sont rarement alimentés en continue via une prise électrique.

C'est pourquoi plusieurs protocoles ont été créé en prenant en compte ces contraintes, l'un d'eux étant \textbf{6LoWPAN}. Il nous a été demandé d'effectuer une étude sur ce protocole car il présente une particularité que son principal concurrent ne possède pas (\textbf{ZigBee}), celle de pouvoir router l'information depuis n'importe quels nœuds (\textit{node} en anglais). Cela ne nous obliges pas à avoir une topologie Maitre/Esclave mais plutôt un réseau maillé ce qui permet de couvrir de plus grande superficie.

Notre projet consista en une étude de 6LoWPAN ainsi qu'à son expérimentation grâce à des cartes achetés par notre tuteur, le but ultime étant de pourvoir router de l'information via n'importe quelle carte. Aussi ce projet était à vocation exploratoire pour préparer d'autres projets sur l'IoT dans les années futures. La question était donc :

\begin{center}
\textbf{\textit{Comment faire router de l'information grâce à 6LoWPAN ?}}
\end{center}

Pour répondre, nous allons, dans un premier temps, vous présenter plus en détails l'IoT, certains protocoles et bien sur 6LowPAN, ensuite nous reviendrons sur le travail technique que nous avons effectué sur les cartes puis nous dresserons un bilan de ce projet.
