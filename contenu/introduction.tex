% Introduction du rapport

L'Internet des Objets (\textbf{IdO} ou \textbf{IoT} pour \textit{Internet of Things} en anglais) représente la jonction entre Internet et le monde des capteurs, qui ne sont généralement pas directement reliés aux équipements actifs des réseaux, comme un routeur. En effet, usuellement les données passent plutôt par une phase de traitement sur un ordinateur avant d'être potentiellement envoyées sur le réseau.

Tous les concepts derrière l'IoT soulèvent plusieurs problématiques telles que l'accroissement exponentiel du volume de données sur le réseau, dû à l'explosion du nombre d'objets connectés, mais aussi des problèmes de nature énergétique, car les capteurs sont rarement alimentés en continu via une prise électrique.

C'est pourquoi plusieurs protocoles ont été créés en prenant en compte ces contraintes, l'un d'eux étant \textbf{6LoWPAN}. Il nous fût demandé d'effectuer une étude sur ce protocole car il présente une particularité que son principal concurrent (\textbf{ZigBee}) ne possède pas, celle de pouvoir router l'information depuis n'importe quel nœud (\textit{node} en anglais). Plutôt qu’une topologie maître/esclave, il permet l’établissement d’un réseau maillé, ce qui permet de couvrir de plus grandes superficies.

Notre projet consista en une étude de 6LoWPAN ainsi qu'à son expérimentation grâce à des cartes achetées par notre tuteur, le but ultime étant de pouvoir router de l'information via n'importe quelle carte. Aussi ce projet était à vocation exploratoire pour préparer d'autres projets sur l'IoT dans les années futures. La question était donc :

\begin{center}
\textbf{\textit{Comment router de l'information grâce à 6LoWPAN ?}}
\end{center}

Pour répondre, nous allons, dans un premier temps, vous présenter plus en détails l'IoT, certains protocoles et bien sûr 6LoWPAN. Ensuite nous reviendrons sur le travail technique que nous avons effectué sur les cartes puis nous dresserons un bilan de ce projet.
