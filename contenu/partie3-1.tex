% Partie 3.1 - Travail accompli %

\subsection{Travail accompli}

Pour la réalisation de ce projet, nous avons clairement défini trois étapes.

% recherche
La première partie fût une phase de recherche et d'apprentissage, nous avons dû nous familiariser avec les concepts de l'IoT ainsi que le protocole 6LoWPAN.
De plus, le fait de pouvoir en apprendre plus sur l'IoT va nous être réellement bénéfique car cette révolution est imminente.

On pourrait même dire qu’elle a commencé avec la banalisation de la domotique et l'arrivée des montres et bracelets connectées qui étendent notre PAN.
La compréhension de ce protocole couvre une partie de ce que nous avons déjà vu en troisième année (notamment IPv6).

% TI
La seconde partie fût celle qui nous amena à travailler avec les kits de développements de TI et \textit{Contiki}. Nous avons tout d'abord testé les cartes avec le logiciel \textit{SmartRF Studio} puis nous avons installé \textit{Contiki}.

Une fois la prise en main de \textit{Contiki} terminée (avec quelque tests sur \textit{Cooja}, le simulateur de transmission), nous avons commencé à développer nos propres programmes pour les cartes. Quand nous avons finalement compris comment envoyer les programmes sur les cartes, nous avons réalisé un premier prototype.

% Arduino
La dernière partie était l'ajout des Arduino au prototype, pour cela nous avons dû utiliser un projet de Télécom Bretagne, qui consistait en une adaptation de \textit{Contiki} pour les Arduino et les modules Xbee.
