\documentclass[a4paper, oneside, 12pt]{article}

%%%%%%%%%% Paquets %%%%%%%%%%

% Paquets pour le Français
\usepackage[utf8]{inputenc} % Gestion encodages
\usepackage[T1]{fontenc} % ???
\usepackage[francais]{babel} % Typographie française

% Divers
\usepackage{bibtopic}
\usepackage[noabbrev]{cleveref}
\usepackage{url}
\usepackage{here}
\usepackage{enumitem}

% Images
\usepackage{graphicx}

% Mise en page
\usepackage[top=2.5cm, bottom=2.5cm, left=2.5cm, right=2.5cm, footskip=4.5cm, includefoot]{geometry}
\parskip=12pt

\usepackage{setspace}
\newenvironment{rmskip}{\bgroup\parskip=8pt\setstretch{1.0}}{\egroup}

\usepackage{color} % Utilisé par \todo

% Tabulations verbatim
% http://www.grappa.univ-lille3.fr/FAQ-LaTeX/6.16.html
\makeatletter
{\catcode`\^^I=\active
\gdef\verbatim{\catcode`\^^I=\active\def^^I{\hspace*{4em}}%
\@verbatim \frenchspacing\@vobeyspaces \@xverbatim}}
\makeatother

%%%%%%%%%% Commandes supplémentaires %%%%%%%%%%%

% Marquage du travail restant
% [#1] Texte à afficher
\newcommand{\todo}[1][Il y a là encore des choses à écrire !]{\colorbox{yellow}{#1}}

%%% Gestion du sommaire
% Section non numérotée
% {#1} Nom de la section
\newcommand{\sectionSpeciale}[1]{\section*{#1}\addcontentsline{toc}{section}{#1}}
% Section non numérotée dont le titre n’est pas affiché.
% {#1} Nom de la section
\newcommand{\sectionCachee}[1]{\addcontentsline{toc}{section}{#1}}

%%% Gestion des annexes
\makeatletter
	
	\newcounter{annctr}
	
	% Définir une annexe
	% {#1} Nom de l’annexe
	\newcommand{\annexe}[1]{\stepcounter{annctr}\addcontentsline{ann}{section}{\protect\numberline {\Alph{annctr}}#1}\section*{\numberline {\Alph{annctr}}#1}}
	% Lister les annexes
	% Note : ce serait quand même bien de le faire fonctionner sans avoir besoin du makefile.
	\newcommand\listeannexes{\sectionSpeciale{Annexes}\input{annexes.tex}}
	
	% Déclaration et ouverture du fichier de stockage de la liste des annexes
	\newwrite\tf@ann
	\immediate\openout\tf@ann\jobname.ann\relax
      
\makeatother

% \newcommand{\bibliographie}[2]{\sectionCachee{#1}\renewcommand{\refname}{#1}\bibliographystyle{plain}\bibliography{#2}}
\newcommand{\bibliographie}[2]{%
\begin{btSect}[plain]{#2}%
	\subsection*{#1}%
	\btPrintAll%/\btPrintCited
\end{btSect}%
}
\newcommand{\bibliographieSansTitre}[1]{%
\begin{btSect}[plain]{#1}%
	\btPrintAll%/\btPrintCited
\end{btSect}%
}

%%%%%%%%%% Abstract %%%%%%%%%%

\newcommand\keywords[2]{%
	\begin{itshape}
		\noindent\textbf{#1} #2
	\end{itshape}
}

%%%%%%%%%% Abstraction de la mise en forme %%%%%%%%%%

\newcommand\nom[1]{\textsc{#1}}
\newcommand\classe[1]{\textit{#1}}

%%%%%%%%%% Outils %%%%%%%%%%

\newcommand\rpFun{{\em{Informatique des Systèmes Embarqués}}}
\newcommand\rpFdeux{{\em{Génie Logiciel et Systèmes Informatiques}}}
\newcommand\rpFtrois{{\em{Systèmes d’Information et Aide à la Décision}}}
\newcommand\rpFquatre{{\em{Calcul et Modélisation Scientifiques}}}
\newcommand\rpFcinq{{\em{Réseaux et télécommunications}}}

\def\rpTitreLexique{Glossaire}
\def\rpTypeTuteurIsima{Responsable}
\def\rpTypeTuteurEntreprise{Responsable}
\def\rpInterligne{1.5}
\def\rpDossier{contenu}
\def\rpLargeurLogoEntreprise{6cm}

% R�tro-compatibilit�
% Ne pas modifier dans le d�p�t ma�tre, pour �viter d'�craser la configuration
% des d�p�ts l'utilisant.
% R�tro-compatibilit�
% Ne pas modifier dans le d�p�t ma�tre, pour �viter d'�craser la configuration
% des d�p�ts l'utilisant.
% R�tro-compatibilit�
% Ne pas modifier dans le d�p�t ma�tre, pour �viter d'�craser la configuration
% des d�p�ts l'utilisant.
\input{contenu/configuration}




\setstretch{\rpInterligne}
